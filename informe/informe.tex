\documentclass[11pt, a4paper, twoside]{article} % o book si es muy extenso
\usepackage[utf8]{inputenc}
\usepackage[T1]{fontenc}
\usepackage{lmodern} % Fuente moderna y legible
\usepackage{microtype} % Mejora el espaciado y la justificación
\usepackage{geometry} % Para controlar los márgenes
    \geometry{
        a4paper,
        top=2.5cm,
        bottom=2.5cm,
        left=3cm,
        right=2.5cm,
        headheight=14pt,
        footskip=1.2cm
    }
\usepackage{fancyhdr} % Para personalizar encabezados y pies de página
\usepackage{graphicx} % Para incluir imágenes
\usepackage{hyperref} % Para enlaces interactivos y metadatos
    \hypersetup{
        colorlinks=true,
        linkcolor=blue,
        filecolor=magenta,
        urlcolor=cyan,
        pdftitle={Título de tu Proyecto},
        pdfauthor={Tu Nombre},
        pdfsubject={Informe de Proyecto Escolar},
        pdfkeywords={Proyecto, Escuela, Informe, LaTeX},
        bookmarks=true,
        bookmarksopen=true,
        bookmarksnumbered=true,
    }
\usepackage{titlesec} % Para personalizar títulos de secciones
\usepackage{enumitem} % Para personalizar listas
\usepackage{setspace} % Para controlar el interlineado
    \onehalfspacing % Interlineado de 1.5 para mejor lectura
\usepackage{listings}
\usepackage{xcolor} % Asegúrate de tenerlo para los colores

% --- Configuración de listings para código ---
\definecolor{codegreen}{rgb}{0,0.6,0}
\definecolor{codegray}{rgb}{0.5,0.5,0.5}
\definecolor{codepurple}{rgb}{0.58,0,0.82}
\definecolor{backcolour}{rgb}{0.95,0.95,0.92}

\lstdefinestyle{mystyle}{
    backgroundcolor=\color{backcolour},
    commentstyle=\color{codegreen},
    keywordstyle=\color{magenta},
    numberstyle=\tiny\color{codegray},
    stringstyle=\color{codepurple},
    basicstyle=\ttfamily\footnotesize, % Fuente monoespaciada, tamaño pequeño
    breakatwhitespace=false,
    breaklines=true,
    captionpos=b,
    keepspaces=true,
    numbers=left, % Números de línea a la izquierda
    numbersep=5pt,
    showspaces=false,
    showstringspaces=false,
    showtabs=false,
    tabsize=2, % Tamaño de la tabulación
    frame=single, % Marco alrededor del código
    rulecolor=\color{black},
    frameround=tttt, % Esquinas redondeadas
    extendedchars=true, % Para caracteres extendidos (como 'ñ')
    literate={á}{{\'a}}1 {é}{{\'e}}1 {í}{{\'i}}1 {ó}{{\'o}}1 {ú}{{\'u}}1 {ñ}{{\~n}}1 % Para manejar tildes y eñes
}

\lstset{style=mystyle} % Aplicar el estilo por defecto
\usepackage{lastpage} % Para obtener el número total de páginas

% --- Personalización de títulos de secciones ---
\titleformat{\section}[block]
  {\normalfont\Large\bfseries\color{green!70!black}}
  {\thesection}{1em}{}
\titlespacing*{\section}
  {0pt}{1.5ex plus .1ex minus .2ex}{1ex plus .1ex}

\titleformat{\subsection}[block]
  {\normalfont\large\bfseries\color{blue!50!black}}
  {\thesubsection}{1em}{}
\titlespacing*{\subsection}
  {0pt}{1.5ex plus .1ex minus .2ex}{1ex plus .1ex}

% --- Configuración de encabezados y pies de página ---
\pagestyle{fancy}
\fancyhf{} % Borra los ajustes predeterminados
\fancyhead[RO,LE]{\thepage} % Número de página a la derecha en páginas impares, a la izquierda en pares
\fancyhead[LO]{\nouppercase{\rightmark}} % Título de la sección en la parte izquierda (páginas impares)
\fancyhead[RE]{\nouppercase{\leftmark}} % Título del capítulo en la parte derecha (páginas pares) (si usas 'book' o 'report')
\fancyfoot[C]{\textbf{Informe de Proyecto}} % Texto en el pie de página centrado
\renewcommand{\headrulewidth}{0.4pt} % Línea en el encabezado
\renewcommand{\footrulewidth}{0.4pt} % Línea en el pie de página

\begin{document}

\thispagestyle{empty} % Para que la primera página no tenga encabezado ni pie de página

\begin{titlepage}
    \begin{center}

        % --- Título del Proyecto ---
        {\color{green!80!black}\Huge\bfseries Compilador de HULK}\par
        \vspace{0.5cm}

        % --- Integrantes del Equipo ---
        {\Large\bfseries Integrantes:}\par
        \vspace{0.3cm} % Espacio entre "Integrantes:" y los nombres
        {\large  Lia S. Lope\'ez Rosales C312}\par
        {\large  Ariadna Vel\'azquez Rey C311}\par
        \vspace{1.5cm} % Espacio después de los nombres

        % --- Información del Curso/Asignatura ---
        {\large Asignatura: Compilaci\'on}\par
        {\large Universidad de La Habana}\par
        \vspace{0.5cm}

        % --- Fecha ---
        {\large \today}\par % Muestra la fecha actual
        \vfill % Empuja el contenido restante hacia abajo


    \end{center}
\end{titlepage}

\clearpage
% ... (el resto de tu documento sigue aquí, como la tabla de contenido y las secciones)

\clearpage % Para empezar el contenido en una nueva página
% Aquí puedes incluir tu tabla de contenido, si la necesitas:
% \tableofcontents
% \clearpage

% --- Comienza el contenido de tu informe ---
\section{Introducción}
Este informe describe el proceso de creación de un compilador para el lenguaje HULK (Havana University Language for Kompilers)
utilizando como lenguaje base C++.

El compilador está diseñado siguiendo una arquitectura modular que divide el proceso de compilación en etapas bien definidas, cada una con responsabilidades específicas y mecanismos de manejo de errores robustos.

\subsection{Flujo del Programa}
El flujo del programa sigue una secuencia clara y estructurada de etapas de compilación. El punto de entrada es la función \texttt{main()} en \texttt{main.cpp}, que coordina todo el proceso:

\begin{enumerate}
    \item \textbf{Inicialización}:
    \begin{itemize}
        \item Procesamiento de argumentos de línea de comandos para determinar el archivo fuente
        \item Apertura y validación del archivo \texttt{.hulk} especificado
        \item Configuración del entorno de compilación
    \end{itemize}

    \item \textbf{Análisis Léxico y Sintáctico}:
    \begin{itemize}
        \item Ejecución del análisis léxico mediante \texttt{yylex()}
        \item Verificación de errores léxicos mediante \texttt{lex\_error}
        \item Ejecución del análisis sintáctico con \texttt{yyparse()}
        \item Construcción del AST durante el proceso
    \end{itemize}

    \item \textbf{Validación del AST}:
    \begin{itemize}
        \item Verificación de la validez del árbol mediante \texttt{is\_valid\_ast()}
        \item Comprobación de nodos nulos y estructura correcta
        \item Impresión de información de diagnóstico sobre los nodos raíz
    \end{itemize}

    \item \textbf{Análisis Semántico}:
    \begin{itemize}
        \item Creación del analizador semántico (\texttt{SemanticAnalyzer})
        \item Ejecución del análisis sobre el AST
        \item Verificación de tipos y reglas semánticas
    \end{itemize}

    \item \textbf{Generación de Código}:
    \begin{itemize}
        \item Inicialización del contexto de generación (\texttt{CodeGenContext})
        \item Generación de código LLVM IR
        \item Escritura del código generado en \texttt{hulk-low-code.ll}
    \end{itemize}

    \item \textbf{Finalización}:
    \begin{itemize}
        \item Liberación de recursos mediante \texttt{delete\_ast()}
        \item Cierre de archivos y limpieza de memoria
        \item Retorno del estado de la compilación
    \end{itemize}
\end{enumerate}

En cada etapa, el compilador implementa un manejo de errores robusto que permite:
\begin{itemize}
    \item Detección temprana de problemas
    \item Mensajes de error descriptivos con información de ubicación
    \item Limpieza apropiada de recursos en caso de fallo
    \item Códigos de retorno específicos para diferentes tipos de error
\end{itemize}


\section{An\'alisis L\'exico: Lexer}

El analizador léxico del compilador \textbf{HULK} está implementado utilizando \textbf{Flex}, con un diseño que prioriza la precisión en el seguimiento de posición y el manejo de errores. La implementación incluye características avanzadas para el seguimiento de ubicación y la detección temprana de errores.

\subsection{Configuración y Características}

El lexer está configurado con las siguientes características principales:

\begin{itemize}
    \item \textbf{Opciones de Flex}:
    \begin{itemize}
        \item \texttt{noyywrap}: Desactiva el procesamiento de múltiples archivos
        \item \texttt{yylineno}: Habilita el seguimiento automático de líneas
        \item \texttt{nounput}: Optimiza el lexer eliminando funciones no utilizadas
    \end{itemize}
    
    \item \textbf{Seguimiento de Posición}:
    \begin{itemize}
        \item Variable \texttt{yycolumn} para seguimiento preciso de columnas
        \item Macro \texttt{YY\_USER\_ACTION} que actualiza automáticamente:
        \begin{itemize}
            \item Línea inicial y final (\texttt{first\_line}, \texttt{last\_line})
            \item Columna inicial y final (\texttt{first\_column}, \texttt{last\_column})
        \end{itemize}
    \end{itemize}
\end{itemize}

\subsection{Tokens y Patrones}

El lexer reconoce los siguientes tipos de tokens:

\begin{itemize}
    \item \textbf{Literales}:
    \begin{itemize}
        \item \textbf{Números}: Enteros y decimales (\texttt{[0-9]+(\textbackslash.[0-9]+)?})
        \item \textbf{Cadenas}: Con soporte para caracteres de escape (\texttt{"(?:\textbackslash\textbackslash.|[]\^{}")*"})
        \item \textbf{Booleanos}: \texttt{True} y \texttt{False}
        \item \textbf{Null}: Valor nulo
    \end{itemize}

    \item \textbf{Operadores}:
    \begin{itemize}
        \item \textbf{Aritméticos}: \texttt{+}, \texttt{-}, \texttt{*}, \texttt{/}, \texttt{\%}, \texttt{\textasciicircum}
        \item \textbf{Comparación}: \texttt{<}, \texttt{>}, \texttt{<=}, \texttt{>=}, \texttt{==}, \texttt{!=}
        \item \textbf{Lógicos}: \texttt{\&}, \texttt{|}, \texttt{!}
        \item \textbf{Concatenación}: \texttt{@}, \texttt{@@}
    \end{itemize}

    \item \textbf{Palabras Clave}:
    \begin{itemize}
        \item \textbf{Control de Flujo}: \texttt{if}, \texttt{else}, \texttt{elif}, \texttt{while}, \texttt{for}
        \item \textbf{Declaraciones}: \texttt{function}, \texttt{let}, \texttt{in}, \texttt{type}
        \item \textbf{POO}: \texttt{new}, \texttt{self}, \texttt{inherits}, \texttt{base}
        \item \textbf{Funciones Matemáticas}: \texttt{sin}, \texttt{cos}, \texttt{max}, \texttt{min}, \texttt{sqrt}, \texttt{exp}, \texttt{log}, \texttt{rand}
    \end{itemize}

    \item \textbf{Identificadores}: Patrón \texttt{[a-zA-Z\_][a-zA-Z0-9\_]*} con validación específica
\end{itemize}

\subsection{Características Especiales}

La implementación incluye funcionalidades adicionales:

\begin{itemize}
    \item \textbf{Gestión de Strings}:
    \begin{itemize}
        \item Eliminación automática de comillas delimitadoras
        \item Preservación de caracteres de escape
    \end{itemize}

    \item \textbf{Depuración}:
    \begin{itemize}
        \item Mensajes informativos para cada token reconocido
        \item Información de posición para facilitar el desarrollo
    \end{itemize}

    \item \textbf{Integración con el Parser}:
    \begin{itemize}
        \item Comunicación mediante \texttt{yylval} para pasar valores
        \item Sincronización de posición mediante \texttt{yylloc}
    \end{itemize}
\end{itemize}

\section{An\'alisis Sint\'actico: Parser}

El analizador sintáctico del compilador \textbf{HULK} está implementado utilizando \textbf{Bison}, con un diseño que prioriza la construcción precisa del AST y el manejo de expresiones complejas. La implementación sigue una estructura clara y modular que facilita la extensión del lenguaje.

\subsection{Estructura del Parser}

El parser está organizado en secciones bien definidas:

\begin{itemize}
    \item \textbf{Declaraciones Iniciales}:
    \begin{itemize}
        \item Estructura \texttt{YYLTYPE} para seguimiento de ubicación
        \item Vector \texttt{root} para almacenar el AST
        \item Funciones auxiliares como \texttt{vectorize} para manejo de argumentos
    \end{itemize}

    \item \textbf{Tipos Semánticos}:
    \begin{itemize}
        \item Tipos básicos: \texttt{num}, \texttt{str}, \texttt{boolean}
        \item Tipos compuestos: \texttt{node}, \texttt{list}, \texttt{param}
        \item Tipos específicos: \texttt{if\_branch}, \texttt{let\_decl}, \texttt{type\_body}
    \end{itemize}
\end{itemize}

\subsection{Tokens y Precedencia}

La gramática define una jerarquía clara de operadores:

\begin{itemize}
    \item \textbf{Operadores de Mayor Precedencia}:
    \begin{itemize}
        \item Multiplicación, División, Módulo (\texttt{MUL}, \texttt{DIV}, \texttt{MOD})
        \item Suma y Resta (\texttt{ADD}, \texttt{SUB})
    \end{itemize}

    \item \textbf{Operadores de Comparación}:
    \begin{itemize}
        \item Relacionales: \texttt{<}, \texttt{>}, \texttt{<=}, \texttt{>=}
        \item Igualdad: \texttt{==}, \texttt{!=}
    \end{itemize}

    \item \textbf{Operadores Lógicos}:
    \begin{itemize}
        \item AND (\texttt{\&}), OR (\texttt{|}), NOT (\texttt{!})
    \end{itemize}

    \item \textbf{Operadores Especiales}:
    \begin{itemize}
        \item Concatenación: \texttt{@}, \texttt{@@}
        \item Funciones matemáticas: \texttt{sin}, \texttt{cos}, \texttt{sqrt}, etc.
    \end{itemize}
\end{itemize}

\subsection{Reglas Gramaticales}

La gramática está estructurada en niveles jerárquicos:

\begin{itemize}
    \item \textbf{Nivel Superior}:
    \begin{itemize}
        \item \texttt{program}: Lista de statements
        \item \texttt{statement}: Expresiones, declaraciones y bloques
    \end{itemize}

    \item \textbf{Expresiones}:
    \begin{itemize}
        \item Literales: números, strings, booleanos, null
        \item Operaciones binarias y unarias
        \item Llamadas a funciones y métodos
        \item Expresiones de control de flujo
    \end{itemize}

    \item \textbf{Estructuras de Control}:
    \begin{itemize}
        \item \texttt{if\_expr}: Soporte para \texttt{if}, \texttt{elif}, \texttt{else}
        \item \texttt{while\_expr}: Bucles while con condición
        \item \texttt{for\_expr}: Bucles for con rangos
    \end{itemize}

    \item \textbf{Programación Orientada a Objetos}:
    \begin{itemize}
        \item Declaración de tipos con \texttt{type\_decl}
        \item Herencia mediante \texttt{inherits}
        \item Métodos y atributos con \texttt{method\_decl} y \texttt{attribute\_decl}
        \item Llamadas a métodos y constructores
    \end{itemize}
\end{itemize}

\subsection{Construcción del AST}

Durante el análisis sintáctico, se construye el AST de manera incremental:

\begin{itemize}
    \item \textbf{Nodos Básicos}:
    \begin{itemize}
        \item Literales y identificadores
        \item Operadores binarios y unarios
        \item Llamadas a funciones
    \end{itemize}

    \item \textbf{Nodos Compuestos}:
    \begin{itemize}
        \item Bloques de código
        \item Estructuras de control
        \item Declaraciones de tipos y funciones
    \end{itemize}

    \item \textbf{Información de Ubicación}:
    \begin{itemize}
        \item Cada nodo almacena su línea de origen (\texttt{yylloc.first\_line})
        \item Facilita el reporte preciso de errores
    \end{itemize}
\end{itemize}


\section{Chequeo Sem\'antico}
El chequeo sem\'antico en el compilador \textbf{HULK} se implementa mediante un sistema sofisticado que utiliza el \textbf{Patr\'on Visitor} para recorrer y analizar el AST. La implementación se divide en varios componentes principales que trabajan en conjunto para garantizar la corrección semántica del programa.

\subsection{Componentes Principales}

\begin{itemize}
    \item \textbf{SemanticAnalyzer}: Implementa el patrón Visitor y coordina todo el análisis semántico.
    \item \textbf{SymbolTable}: Gestiona los símbolos y ámbitos del programa.
    \item \textbf{FunctionCollector}: Recolecta y analiza las declaraciones de funciones.
\end{itemize}

\subsection{Tabla de S\'imbolos}

La tabla de símbolos (\texttt{SymbolTable}) es una estructura sofisticada que maneja:

\begin{itemize}
    \item \textbf{\'Ambitos Anidados}:
    \begin{itemize}
        \item Vector de mapas para manejar ámbitos (\texttt{scopes})
        \item Métodos \texttt{enterScope()} y \texttt{exitScope()} para gestión de ámbitos
        \item Búsqueda de símbolos en ámbitos anidados
    \end{itemize}

    \item \textbf{Tipos de S\'imbolos}:
    \begin{itemize}
        \item Variables con tipo y estado de constante
        \item Funciones con tipo de retorno y parámetros
        \item Tipos definidos por el usuario con atributos y métodos
    \end{itemize}

    \item \textbf{Tipos Predefinidos}:
    \begin{itemize}
        \item Object, Number, String, Boolean, Null
        \item Jerarquía de tipos y relaciones de subtipado
    \end{itemize}
\end{itemize}

\subsection{An\'alisis de Tipos}

El sistema implementa un sofisticado análisis de tipos que incluye:

\begin{itemize}
    \item \textbf{Inferencia de Tipos}:
    \begin{itemize}
        \item Análisis de uso de parámetros en el cuerpo de funciones
        \item Inferencia basada en operaciones y contexto
        \item Resolución de tipos más específicos comunes
    \end{itemize}

    \item \textbf{Verificaci\'on de Operaciones}:
    \begin{itemize}
        \item Operaciones aritméticas (\texttt{+}, \texttt{-}, \texttt{*}, \texttt{/}, \texttt{\%})
        \item Operaciones de comparación (\texttt{>}, \texttt{<}, \texttt{>=}, \texttt{<=})
        \item Operaciones lógicas (\texttt{\&}, \texttt{|}, \texttt{!})
        \item Concatenación de strings (\texttt{@}, \texttt{@@})
    \end{itemize}

    \item \textbf{Jerarqu\'ia de Tipos}:
    \begin{itemize}
        \item Verificación de conformidad de tipos (\texttt{conformsTo})
        \item Búsqueda del ancestro común más bajo (\texttt{lowestCommonAncestor})
        \item Manejo de herencia
    \end{itemize}
\end{itemize}

\subsection{Validaci\'on Sem\'antica}

El analizador semántico realiza múltiples validaciones:

\begin{itemize}
    \item \textbf{Declaraciones}:
    \begin{itemize}
        \item Unicidad de nombres en el ámbito actual
        \item Tipos correctos en declaraciones
    \end{itemize}

    \item \textbf{Funciones}:
    \begin{itemize}
        \item Compatibilidad de tipos en argumentos
        \item Inferencia de tipos de retorno
        \item Validación de funciones matemáticas built-in
    \end{itemize}

    \item \textbf{Programaci\'on Orientada a Objetos}:
    \begin{itemize}
        \item Declaración y uso de tipos
        \item Validación de herencia
        \item Llamadas a métodos y constructores
        \item Acceso a atributos
    \end{itemize}

    \item \textbf{Control de Flujo}:
    \begin{itemize}
        \item Tipos correctos en condiciones
        \item Validación de bucles
        \item Comprobación de expresiones let-in
    \end{itemize}
\end{itemize}


\section{Generación de Código LLVM}

La generación de código intermedio en el compilador \textbf{Hulk} se realiza utilizando \textbf{LLVM IR} (Intermediate Representation). El sistema está diseñado de manera modular, con una clara separación de responsabilidades entre la generación de código, el manejo de tipos y el contexto de generación.

\subsection{Componentes Principales}

La implementación se divide en tres componentes principales:

\begin{itemize}
    \item \textbf{LLVMGenerator}: Implementa el patrón Visitor para recorrer el AST y generar código LLVM IR.
    \item \textbf{CodeGenContext}: Encapsula todo el estado y contexto necesario para la generación de código.
    \item \textbf{TypeSystem}: Maneja el sistema de tipos, incluyendo definiciones de tipos y sus instancias.
\end{itemize}

\subsection{Estructura del Generador}

El generador de código (\texttt{LLVMGenerator}) implementa el patrón Visitor y proporciona métodos específicos para cada tipo de nodo del AST:

\begin{itemize}
    \item Manejo de literales (números, booleanos, strings)
    \item Operaciones binarias y unarias
    \item Funciones built-in y definidas por el usuario
    \item Estructuras de control (if, while, for)
    \item Soporte para POO (declaración de tipos, instanciación, llamadas a métodos)
\end{itemize}

\subsection{Contexto de Generación}

El \texttt{CodeGenContext} mantiene el estado durante la generación de código:

\begin{itemize}
    \item \textbf{Estado LLVM}: Contexto global, builder y módulo
    \item \textbf{Gestión de Ámbitos}: Pilas para variables locales y funciones
    \item \textbf{Sistema de Tipos}: Integración con el sistema de tipos para POO
    \item \textbf{Pila de Valores}: Mecanismo para pasar valores entre nodos durante el recorrido del AST
\end{itemize}

\subsection{Sistema de Tipos}

El \texttt{TypeSystem} proporciona soporte completo para POO:

\begin{itemize}
    \item Definición de tipos con atributos y métodos
    \item Soporte para herencia
    \item Gestión de instancias y sus variables
    \item Manejo de constructores y llamadas base
\end{itemize}

\subsection{Flujo de Generación}

El proceso de generación sigue estos pasos:

\begin{enumerate}
    \item \textbf{Inicialización}:
    \begin{itemize}
        \item Configuración del contexto LLVM y builder
        \item Declaración de funciones externas (printf, malloc, operaciones matemáticas)
        \item Registro de tipos y funciones del usuario
    \end{itemize}
    
    \item \textbf{Generación de Código}:
    \begin{itemize}
        \item Recorrido del AST usando el patrón Visitor
        \item Generación de instrucciones LLVM para cada tipo de nodo
        \item Manejo de la pila de valores para comunicación entre nodos
    \end{itemize}
    
    \item \textbf{Gestión de Memoria}:
    \begin{itemize}
        \item Uso de \texttt{alloca} para variables locales
        \item Manejo de memoria para strings y objetos
        \item Gestión de ámbitos anidados
    \end{itemize}
    
    \item \textbf{Verificación}:
    \begin{itemize}
        \item Validación del módulo LLVM generado
        \item Generación del archivo IR final
    \end{itemize}
\end{enumerate}

\subsection{Ejemplo de Generación}

Consideremos la generación de código para una operación binaria:

\begin{enumerate}
    \item El visitor procesa recursivamente los operandos izquierdo y derecho
    \item Los valores resultantes se obtienen de la pila de valores
    \item Se genera la instrucción LLVM correspondiente según el operador
    \item El resultado se coloca en la pila para uso posterior
\end{enumerate}

Por ejemplo, para la expresión \texttt{a + b} donde ambos son números:

\begin{itemize}
    \item Se generan las cargas de las variables \texttt{a} y \texttt{b}
    \item Se crea una instrucción \texttt{fadd} usando el builder
    \item El resultado se almacena en la pila de valores
\end{itemize}

\section{Manejo de Errores}

El manejo de errores en el compilador \textbf{Hulk} está implementado de manera modular y robusta, con mecanismos específicos en cada fase de la compilación. El sistema está diseñado para detectar y reportar errores de manera precisa, proporcionando información detallada sobre la ubicación y naturaleza de cada error.

\subsection{Errores Léxicos}

El analizador léxico, implementado con \textbf{Flex}, incluye un sistema de seguimiento de posición preciso:

\begin{itemize}
    \item \textbf{Seguimiento de Posición}: Se mantiene un registro exacto de línea y columna para cada token mediante las variables \texttt{yylineno} y \texttt{yycolumn}.
    \item \textbf{Detección de Errores}:
    \begin{itemize}
        \item Identificadores inválidos (comenzando con guiones bajos o números)
        \item Caracteres no reconocidos en el lenguaje
        \item Tokens malformados
    \end{itemize}
    \item \textbf{Reporte de Errores}: Los errores se reportan inmediatamente con:
    \begin{itemize}
        \item Descripción del error
        \item Línea y columna exacta del error
        \item Token problemático
    \end{itemize}
\end{itemize}

\subsection{Errores Sintácticos}

El parser, construido con \textbf{Bison}, implementa un sistema de recuperación de errores:

\begin{itemize}
    \item \textbf{Sincronización}: Uso de tokens de sincronización (como el punto y coma) para recuperarse de errores.
    \item \textbf{Localización}: Aprovecha la información de ubicación (\texttt{YYLTYPE}) para reportar errores con precisión.
    \item \textbf{Validación de AST}: Verificación de la validez del árbol sintáctico generado:
    \begin{itemize}
        \item Comprobación de nodos nulos
        \item Validación de estructura del árbol
        \item Verificación de completitud
    \end{itemize}
\end{itemize}

\subsection{Errores Semánticos}

El analizador semántico implementa un sistema sofisticado de detección y reporte de errores:

\begin{itemize}
    \item \textbf{Estructura de Error}:
    \begin{itemize}
        \item Clase \texttt{SemanticError} para encapsular errores
        \item Información de línea y mensaje descriptivo
        \item Acumulación de errores para reporte completo
    \end{itemize}

    \item \textbf{Tipos de Errores}:
    \begin{itemize}
        \item Errores de tipo en operaciones
        \item Variables no declaradas o mal utilizadas
        \item Errores en llamadas a funciones
        \item Problemas de herencia y tipos
    \end{itemize}

    \item \textbf{Recuperaci\'on}:
    \begin{itemize}
        \item Continuación del análisis tras errores
        \item Inferencia de tipos en casos ambiguos
        \item Manejo de tipos desconocidos
    \end{itemize}
\end{itemize}


\subsection{Errores en Generación de Código}

La fase de generación de código LLVM incluye:

\begin{itemize}
    \item \textbf{Manejo de Excepciones}: Captura y manejo de errores durante la generación.
    \item \textbf{Validación de IR}: Verificación del código LLVM generado.
    \item \textbf{Reporte de Errores}: Mensajes detallados sobre problemas en la generación.
\end{itemize}

\subsection{Integración en el Flujo Principal}

El \texttt{main.cpp} coordina el manejo de errores entre las diferentes etapas:

\begin{itemize}
    \item \textbf{Verificación Secuencial}:
    \begin{itemize}
        \item Validación de apertura de archivo
        \item Comprobación de errores léxicos
        \item Verificación de errores sintácticos
        \item Validación del AST generado
        \item Control de errores semánticos
        \item Manejo de errores en generación de código
    \end{itemize}
    \item \textbf{Limpieza de Recursos}: Liberación apropiada de memoria y recursos en caso de error.
    \item \textbf{Códigos de Retorno}: Uso de códigos de salida para indicar el tipo de error encontrado.
\end{itemize}

\end{document}